% Options for packages loaded elsewhere
\PassOptionsToPackage{unicode}{hyperref}
\PassOptionsToPackage{hyphens}{url}
%
\documentclass[
]{article}
\usepackage{amsmath,amssymb}
\usepackage{lmodern}
\usepackage{ifxetex,ifluatex}
\ifnum 0\ifxetex 1\fi\ifluatex 1\fi=0 % if pdftex
  \usepackage[T1]{fontenc}
  \usepackage[utf8]{inputenc}
  \usepackage{textcomp} % provide euro and other symbols
\else % if luatex or xetex
  \usepackage{unicode-math}
  \defaultfontfeatures{Scale=MatchLowercase}
  \defaultfontfeatures[\rmfamily]{Ligatures=TeX,Scale=1}
\fi
% Use upquote if available, for straight quotes in verbatim environments
\IfFileExists{upquote.sty}{\usepackage{upquote}}{}
\IfFileExists{microtype.sty}{% use microtype if available
  \usepackage[]{microtype}
  \UseMicrotypeSet[protrusion]{basicmath} % disable protrusion for tt fonts
}{}
\makeatletter
\@ifundefined{KOMAClassName}{% if non-KOMA class
  \IfFileExists{parskip.sty}{%
    \usepackage{parskip}
  }{% else
    \setlength{\parindent}{0pt}
    \setlength{\parskip}{6pt plus 2pt minus 1pt}}
}{% if KOMA class
  \KOMAoptions{parskip=half}}
\makeatother
\usepackage{xcolor}
\IfFileExists{xurl.sty}{\usepackage{xurl}}{} % add URL line breaks if available
\IfFileExists{bookmark.sty}{\usepackage{bookmark}}{\usepackage{hyperref}}
\hypersetup{
  pdftitle={Péndulo doble},
  hidelinks,
  pdfcreator={LaTeX via pandoc}}
\urlstyle{same} % disable monospaced font for URLs
\usepackage[margin=1in]{geometry}
\usepackage{graphicx}
\makeatletter
\def\maxwidth{\ifdim\Gin@nat@width>\linewidth\linewidth\else\Gin@nat@width\fi}
\def\maxheight{\ifdim\Gin@nat@height>\textheight\textheight\else\Gin@nat@height\fi}
\makeatother
% Scale images if necessary, so that they will not overflow the page
% margins by default, and it is still possible to overwrite the defaults
% using explicit options in \includegraphics[width, height, ...]{}
\setkeys{Gin}{width=\maxwidth,height=\maxheight,keepaspectratio}
% Set default figure placement to htbp
\makeatletter
\def\fps@figure{htbp}
\makeatother
\setlength{\emergencystretch}{3em} % prevent overfull lines
\providecommand{\tightlist}{%
  \setlength{\itemsep}{0pt}\setlength{\parskip}{0pt}}
\setcounter{secnumdepth}{-\maxdimen} % remove section numbering
\ifluatex
  \usepackage{selnolig}  % disable illegal ligatures
\fi

\title{Péndulo doble}
\author{}
\date{\vspace{-2.5em}4/1/2021}

\begin{document}
\maketitle

\emph{Falta escribir una introducción} Antes de estudiar el péndulo
doble comencemos por estudiar el péndulo simple.

\hypertarget{puxe9ndulo-simple}{%
\section{Péndulo simple}\label{puxe9ndulo-simple}}

El péndulo simple consiste en un punto de masa fijado a una estructura
por medio de una vara. Denotamos por \(\theta\) el ángulo formado entre
la vara y el eje vertical. Para simplificar las cuentas, supongamos que
el largo de la vara es 1 y la masa del objeto también es 1. Supongamos
que podemos ignorar la masa de la vara y en un principio también la
fricción (después incluiremos la fricción). Entonces la única fuerza que
actúa sobre el objeto es la gravedad, por lo que la aceleración del
ángulo \(\theta\) es proporcional a \(sin(\theta)\). Podemos representar
el sistema por medio de la siguiente ecuación:

\[
\ddot{\theta} = -g \sin(\theta)
\]

Añadiendo una variable para la velocidad angular
\(\omega = \dot{\theta}\) reescribimos el sistema como:

\[
\begin{aligned}
\dot{\theta} & = \omega \\
\dot{\omega} & = - g \sin(\theta)
\end{aligned}
\]

Este es un sistema conservativo por lo que podemos utilizar el método
del potencial para esbozar su retrato fase.

Primero encontramos la energía potencial del sistema:

\[
\begin{aligned}
u(\theta) & = - \int_{0}^{\theta} -g \sin(t) \,dt = -g \left(\cos(t)\right)\Big|_{0}^{\theta} = g \left(1 - \cos(\theta)\right) \\
u^{\prime}(\theta) & = g \sin(\theta) = 0 \iff \theta = k \pi \quad con \quad k \in \mathbb{Z}
\end{aligned}
\]

Por lo tanto los puntos de equilibrio son de la forma
\((\theta, \omega) = (k \pi, 0) \; con \; k \in \mathbb{Z}\)

Para cada nivel de energía \(h\) tenemos que las órbitas del sistema
están dadas por:

\[
\omega = \pm \sqrt{2\left(h-g\left(1-\cos(\theta)\right)\right)}
\]

Graficando para cada nivel de energía podemos esbozar el retrato fase:

\% Agregar interpretación de las soluciones.

Ahora vamos a empezar a tomar en cuenta la fricción. Supongamos que ésta
es proporcional a la velocidad \(\omega\). Tenemos que agregar un
término a la ecuación de \(\dot{\omega}\) de la forma \(-b \omega\) con
\(b\) una constante positiva. El sistema resultante es:

\[
\begin{aligned}
\dot{\theta} & = \omega \\
\dot{\omega} & = -b \, \omega - g \sin(\theta)
\end{aligned}
\]

Los puntos de equilibrio son donde \(\omega = 0\) y
\(\sin(\theta) = 0\). Es decir, son los puntos de la forma
\((\theta, \omega) = (k \pi, 0) \; con \; k \in \mathbb{Z}\) (siguen
siendo los mismos de antes).

Vamos a analizar el comportamiento cerca de los puntos de equilibrio
utilizando la Jacobiana del sistema (y el teorema de Hartman--Grobman).

\[
Df(\theta, \omega) = \begin{pmatrix}
0 & 1 \\
-g \cos(\theta) & -b
\end{pmatrix}
\]

Para los puntos de equilibrio con un múltiplo par de \(\pi\) tenemos:

\[
Df(2k\pi, 0) = \begin{pmatrix}
0 & 1 \\
-g  & -b
\end{pmatrix}
\]

Su polinomio característico y valores propios:

\[
\begin{aligned}
    p(\lambda) & = \lambda^2 + b \lambda + g \\
    p(\lambda) & = 0 \iff \lambda = \frac{-b \pm \sqrt{b^2-4g}}{2}
\end{aligned}
\]

Dependiendo del signo de \(b^2-4g\) vamos a tener valores propios reales
o complejos. Si asumimos que la fuerza de fricción no es demasiado
fuerte y por lo tanto la constante \(b\) no es muy grande, vamos a tener
doe valores propios complejos con parte real negativa, por lo que el
equilibrio va a ser espiral estable.

Para los puntos de equilibrio que son un múltiplo impar de \(\pi\)
tenemos:

\[
Df((2k+1)\pi, 0) = \begin{pmatrix}
0 & 1 \\
g  & -b
\end{pmatrix}
\]

Su polinomio característico y valores propios:

\[
\begin{aligned}
    p(\lambda) & = \lambda^2 + b \lambda - g \\
    p(\lambda) & = 0 \iff \lambda = \frac{-b \pm \sqrt{b^2+4g}}{2}
\end{aligned}
\]

Podemos ver que ambos valores propios son reales, uno positivo y uno
negativo, por lo tanto tenemos un punto silla.

Para esbozar el retrato fase utilizaremos las ceroclinas.

\[
\begin{aligned}
    \dot{\theta} &= 0 \iff \omega = 0 \\
    \dot{\omega} &= -b \, \omega - g \sin(\theta) = 0  \iff \omega = - \frac{g}{b} \sin(\theta)
\end{aligned}
\]

Utilizando la información que tenemos hasta ahora podemos esbozar el
retrato fase:

Como podemos ver, el retrato fase es parecido al del péndulo sin
fricción. La principal diferencia es que ahora hay espirales en lugar de
las órbitas periódicas de antes. Esto se debe a que, debido a la
frición, el péndulo ya no oscila indefinidamente, sino que cada vez la
trayectoria es más corta y se va acercando poco a poco a la posición de
equilibrio en la que el péndulo cuelga hacia abajo sin moverse (puntos
de la forma \((\theta, \omega) = (2k\pi, 0)\)).

\hypertarget{puxe9ndulo-doble}{%
\section{Péndulo doble}\label{puxe9ndulo-doble}}

Consideremos ahora un péndulo que está unido a otro péndulo. Tenemos una
masa \(m_1\) unida por un una vara de longitud \(l_1\) que, a su vez,
tiene otra masa \(m_2\) unida por medio de una vara de longitud \(l_2\).
Sean \(\theta_1\) y \(\theta_2\) los ángulos que forman cada una de
estas varas con el eje vertical (ver figura).

Este sistema no es tan sencillo. Para plantear las ecuaciones de este
sistema necesitamos utilizar la ecuación de Euler-Lagrange.

Dadas la energía potencial \(V\) de un sistema y su energía cinética
\(T\), definimos el Lagrangiano como:

\[
L = T - V
\]

La ecuación de Euler-Lagrange nos dice:

\[
    \frac{d}{dt} \left( \frac{\partial L}{\partial \dot{x}} \right) = \left( \frac{\partial L}{\partial x} \right)
\]

Para aplicarlo en nuestro caso tenemos que empezar por encontrar las
coordenadas de cada uno de los puntos de masa:

\[
\begin{aligned}
    x_1 &= l_1 \sin(\theta_1) \\
    y_1 &= - l_1 \cos(\theta_1) \\
    x_2 &= l_1 \sin(\theta_1) + l_2 \sin(\theta_2) \\
    y_2 &= - l_1 \cos(\theta_1) - l_2 \cos(\theta_2) \\
\end{aligned}
\]

La energía potencial del sistema está dada por:

\[
\begin{aligned}
    V &= m_1 g y_1 + m_2 g y_2 \\
    &= -(m_1+m_2) g l_1 \cos(\theta_1) - m_2 g l_2 \cos(\theta_2)
\end{aligned}
\]

Para encontrar la energía cinética necesitamos primero los cuadrados de
las velocidades \(v_1\) y \(v_2\):

\[
\begin{aligned}
    v_1^2 &= l_1^2 \dot{\theta}_1^2 \\
    v_2^2 &= l_1^2 \dot{\theta}_1^2 + l_2^2 \dot{\theta}_2^2 + 2 l_1 l_2 \dot{\theta}_1 \dot{\theta}_2 (\cos(\theta_1) \cos(\theta_2) + \sin(\theta_1) \sin(\theta_2)
\end{aligned}
\]

Entonces la energía cinética está dada por:

\[
\begin{aligned}
    T &= \frac{1}{2} m_1 v_1^2 + \frac{1}{2} m_2 v_2^2 \\
    &= \frac{1}{2} m_1 l_1^2 \dot{\theta}_1^2 + \frac{1}{2} m_2 [l_1^2 \dot{\theta}_1^2 + l_2^2 \dot{\theta}_2^2 + 2 l_1 l_2 \dot{\theta}_1 \dot{\theta}_2 \cos(\theta_1 - \theta_2)]
\end{aligned}
\]

El Lagrangiano es:

\[
    L = T - V = \frac{1}{2} (m_1+m_2) l_1^2 \dot{\theta}_1^2 + \frac{1}{2} m_2 l_2^2 \dot{\theta}_2^2 + m_2 l_1 l_2 \dot{\theta}_1 \dot{\theta}_2 \cos(\theta_1 - \theta_2) + (m_1+m_2) g l_1 \cos(\theta_1) + m_2 g l_2 \cos(\theta_2)
\]

Para \(\theta_1\):

\[
\begin{aligned}
    \frac{\partial L}{\partial \dot{\theta_1}} &= m_1 l_1^2 \dot{\theta}_1 + m_2 l_1^2 \dot{\theta}_1 + m_2 l_1 l_2 \dot{\theta}_2 \cos(\theta_1 - \theta_2) \\
    \frac{d}{dt} \left( \frac{\partial L}{\partial \dot{\theta_1}} \right) &=  (m_1+m_2) l_1^2 \ddot{\theta}_1 + m_2 l_1 l_2 \ddot{\theta}_2 \cos(\theta_1 - \theta_2) - m_2 l_1 l_2 \dot{\theta}_2 \sin(\theta_1 - \theta_2) (\dot{\theta}_1 - \dot{\theta}_2) \\
    \frac{\partial L}{\partial \theta_1} &= - l_1 g (m_1+m_2) \sin(\theta_1) - m_2 l_1 l_2 \dot{\theta}_1 \dot{\theta}_2 \sin(\theta_1 - \theta_2) 
\end{aligned}
\]

Por lo tanto la ecuación de Euler-Lagrange para \(\theta_1\) es:

\[
    (m_1 + m_2) l_1^2 \ddot{\theta}_1 + m_2 l_1 l_2 \ddot{\theta} \cos(\theta_1 - \theta_2) + m_2 l_1 l_2 \dot{\theta}_2^2 \sin(\theta_1 - \theta_2) + l_1 g (m_1+m_2) \sin(\theta_1) = 0
\]

Podemos hacer lo mismo para \(\theta_2\) y la ecuación de Euler-Lagrange
queda:

\[
    m_2 l_2^2 \ddot{\theta}_2 + m_2 l_1 l_2 \dot{\theta}_1^2 \cos(\theta_1 - \theta_2) - m_2 l_1 l_2 \dot{\theta}_1^2 \sin(\theta_1 - \theta_2) + l_2 m_2 g \sin(\theta_2) = 0
\]

\hypertarget{animaciones}{%
\section{Animaciones}\label{animaciones}}

\includegraphics{https://media.giphy.com/media/xUOxf7XfmpxuSode1O/giphy.gif}

\end{document}
